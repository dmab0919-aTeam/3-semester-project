Som nævnt i Kapitel ??, bliver dette projekt udviklet agilt. Et af kendetegnene ved agile udviklingsmetoder er at 
planlægning foregår undervejs i forløbet i modsætning til for eksempel vandfaldsmetoden, hvor planlægning er en 
forudsætning for arbejdets påbegyndelse. På baggrund af dette, findes det relevant at bruge tid på at diskutere 
risikohåndtering. Risikohåndtering består af 4 punkter:

\begin{itemize}
    \item Identifikation af risici
    \item Tildeling af sandsynlighed
    \item Evaluering af konsekvenser
    \item Hvordan skal risici håndteres?
\end{itemize}

Formålet med dette kapitel er at diskutere disse 4 punkter, og redegøre for hvordan gruppen har håndteret risikoanalyse/håndtering.

Risk register mangler. Kan ses i Word-filen ScrumBS på OneDrive.