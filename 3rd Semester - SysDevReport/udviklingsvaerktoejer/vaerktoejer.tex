\section{Scrum}\label{sec:scrum}
Scrum er et framework til at udarbejde, vedligeholde komplekse produkter og løbende levere 
værdi til en kunde gennem inkrementelle og iterative processer/teknikker. \\

Dette agile framework har været anvendt til at udarbejde komplekse produkter siden 1990’erne. \\

Frameworket anses for at være let forståeligt, simpelt sammensat, men svær at mestre.

\subsection{Roller}
\subsubsection{Product Owner}
De primære ansvarsområder en Product Owner påtager sig er, at sikre teamet leverer så 
meget værdi til kunden som muligt. Dette sikres gennem vedligeholdelse af Product Backlog’en. 
Den forretning eller virksomhed, som har bestilt et produkt bliver repræsenteret gennem 
Product Owner, som i den forbindelse sikre at teamet arbejder på at levere de funktionaliteten, 
som er vigtigst for kunden. Kort fortalt Stakeholders, Releases og Backlog er hovedområder for en Product Owner.

\subsubsection{Scrum Master}
Det primære ansvar som en Scrum Master er at holde overblikket over teamet og sikre, at Scrum bliver anvendt 
og udført korrekt. I praksis betyder dette at hjælpe Product Owner’en med at definere værdi for kunden, 
Dev Team’et at levere mest værdi og selve Scrum Team’et samlet skal blive bedre for hvert sprint. 
Det er vigtigt at medtage, at en Scrum Master ikke er en chef eller autoritet, men kan i nogle eksempler 
være en udvikler, som enhver anden fra Dev Team’et.

\subsubsection{Dev Team}
Udviklingsteamet er de mennesker som normalt udarbejder selve produktet, altså udviklere, designere 
eller ingeniører osv. Teamet arbejdet som en enhed, regulerer selv sine opgaver, udviklingsmetoder og 
her handler det om, at alle hjælper alle med at opnå de fælles mål for et givet sprint. Det vil sige 
at teamet har mulighed for at træffe egne valg og ikke behøver indrage hverken Scrum Master’en eller 
Product Owner’en i beslutningen, når det har noget med udvikling af gøre.

\subsection{Artefakter}
\subsubsection{Produck Backlog}
Er en liste af user stories I ordnet rækkefølge efter, hvilken værdi det bringer kunden. 
Det er den eneste liste af krav til et givet projekt. Denne liste vedligeholdes og udarbejdes af 
Product Owner’en. En backlog er ikke en låst liste, men en liste der hele tiden kan ændres sig, 
tilføje nye elementer til eller fjerne dem, alt efter hvad der bringer kunden værdi. 

\subsubsection{Sprint Backlog}
Er en anden liste som indeholder de stories og tasks som er udvalgt til et givet sprint. Denne indeholder
ydermere nogle mål for sprintet. 

\subsection{Ceremonier}
\subsubsection{Sprint}
Er en tidsperiode hvori et potentielt “færdigt” og produktionsklart produkt Increment bliver udarbejdet. 
Et Increment er en sum af alle de udvalgte user stories fra backlogen. Et sprint vil normalt ligge mellem 1 til 4 uger 
afhængig af projektet størrelse og firmaets ressourcer.

\subsubsection{Sprint planning}
Er et “møde” hvori gruppen/teamet udarbejder en plan for det kommende sprint. Dette møde må ikke vare mere end 8 timer. 
Her giver “Dev Teamet” sit forslag til hvilke eller hvor mange stories de kan nå at udarbejde, beregnet ud fra deres velocity 
og med fokus på hvor meget de nåede i sidste sprint.

\subsubsection{Daily standup meeting}
Er et lille “møde” hvor teamet er samlet og besvare 3 spørgsmål. Første spørgsmål er “Hvad gjorde folk i går” 
altså hvad arbejder de på. Det næste er “Hvad vil du lave i dag?” hvilket giver teamet mulighed for at få en fælles 
forståelse for hvad alle laver. Til sidst “Er der noget som forhindrer dig i at udføre dit arbejde i dag?”. Det kunne være viden, 
redskaber eller andet som et medlem mangler for at gøre sit arbejde.

\subsubsection{Sprint review}
Er en “demo” visning af selve systemet for hele teamet samt så mange udefrakommende som muligt. Ofte er kunden også til stede ved dette review.

\subsubsection{Sprint retrospective}
Denne ceremoni handler om indsigt, forbedring og hvordan det sidste sprint har forløbet sig.

\section{XP}\label{sec:xp}
XP er en agil udviklingsmetode, brugt til at fremstille software. Metoden byder på 15 regler som skal følges. XP handler om at arbejde i det ekstreme, 
hvis noget virker skal man gøre til det dets ekstreme mulighed.  \\

XP har 5 kerne værdier; Respect, Communication, Simplicity, Feedback og Courage. Hertil 12 praktikker som skal følges og efterleves til 
ekstrem hvis metoden skal anvendes korrekt. \\

Scrum er som beskrevet tidligere i afsnittet et framework til udvikling af produkter, ofte software løsninger, hvilket fungerer som en 
container for mange virksomheder, hvor andre praktikker kan tilføjes til. XP er i den forbindelse en af de praktikker som ofte tilføjes til 
Scrum eller nogle af dets regler. \\

Nogle af XP’s regler er svære at overholde og reglerne anses for at være uden undtagelse, altså alle regler og praktikker skal anvendes til 
det ekstreme. Derimod er nogle af XP’s praktikker en god supplering til Scrum frameworket og det kan hjælpe et nyt, eller uerfaren team til at 
udvikle sig indenfor det agile univers og Scrum frameworket. \\

Det kan endvidere udledes at XP og agil udvikling er lidt et Paradox fordi et team har et mål om at være helt selv organiserende men får 
samtidigt at vide at de “skal” havde en bestemt længde arbejdsuge, programmere sammen 2 og 2 og implementere “Test Driven Development”. 

\section{Kanban}\label{sec:kanban}
Denne udviklings metode er bygget på at visualisere og skabe flow i arbejdet for det given team. Kanban giver mulighed for hurtigt at 
genprioritere arbejdsopgaver og omstillingspararthed i forbindelse med pludselige ændringer.  \\

Kanban er en agil udviklings metode til en virksomhed eller team, som er ny til agil udvikling.

\section{Metodevalg}\label{sec:valgafvaektoej}
