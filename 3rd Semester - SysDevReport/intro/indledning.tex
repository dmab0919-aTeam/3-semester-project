\chapter{Indledning}\label{ch:indledning}
Siden 1970'erne, har softwareudviklere haft for øje at optimere den proces, det er at udvikle nyt software.
Dette har resulteret i et bredt udvalg af udviklingsmetoder/frameworks. Ethvert udviklingsteam bør nu, 
i starten af en udviklingsperiode, besvare dette spørgsmål: "Hvordan arbejder vi?". 
Hver enkelt udviklingsmetode har fordele og ulemper, og kan alle bruges i hver sin situation. \\

Samme spørgsmål har gruppen bag denne rapport, stillet sig selv i starten af hver projektperiode. 
I dette semester har en større del af undervisningen
været centreret omkring procesmodeller og udviklingsmetoder. 
Et af projektets formål er at udvikle en web-applikation, der opfylder studieordningens krav
i forhold til programmering og teknologi. Et andet fokus for dette semester, 
som denne rapport vil dokumentere, er at vælge og følge en specifik udviklingsmetode, 
reflektere over den, og benytte den i praksis ved brug af de rette værktøjer. \\

Før 1990, fulgte de fleste udviklingsmetoder vandfaldsmodellen. En model der ikke er iterativ, 
men strengt plandrevet. I 1990'erne, opstod flere modsvar til
disse metoder, i form af iterative metoder. Blandt disse findes, Unified Process, Scrum og Extreme Programming, 
som alle stadig benyttes i moderne tid.
I 2001 mødtes 17 kendte softwareudviklere på et hotel i Utah, 
hvor de udgav \textit{Det Agile Manifest} som fastslog de centrale værdier i agil udvikling. \\

Denne rapport beskriver hvilke overvejelser en projektgruppe på 
datamatikeruddannelsens 3. semester har gjort sig, med hensyn til valg af udviklingsmetode.
Desuden vil det blive beskrevet hvordan gruppen har arbejdet agilt, 
og hvordan gruppen har håndteret risici og kvalitetssikring.

\newpage
\section{Problemformulering}
Forud for projektarbejdets begyndelse, har projektgruppen på baggrund af brainstorms og kravene i
studieordningen, udformet følgende systemvision og problemformulering. \\

\textbf{Problem statement} \\
"How can the team design and implement a new universal booking system dedicated to movie theaters, 
that solves modern booking problems within this demanding industry?" \\

\textbf{Sub questions} \\
"How can the team through a system development method optimize the proccess and mimimize errors along the way?" \\

"How can a software development approach improve the quality and produced work of the development team throughout this project?"

\section{Systemvision}
Eftersom der i dette semester projekt, ikke er tilknyttet en virksomhed, har gruppen selv udarbejdet en systemvision.
Visionen har til formål at give indblik i biografsystemets formål, og er formuleret som følgende:\\

For members of the movie loving community, who appreciate having a collective application for booking 
in multiple locations and avoiding the disapointment of double booking. We have dedicated our 
time to develop the perfect platform to book movies and to offer a unique user experience and private user function. 
We strive to evolve and keep up with modern tecnology and demands from the customers. \\


% Here is the introduction. \cite{Mittelbach2005} \\

% Bilag kan ses i Bilag \ref{app:sprint2.tex}.