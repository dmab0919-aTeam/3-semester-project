\cleardoublepage
{\selectlanguage{danish}
\pdfbookmark[0]{Danish title page}{label:titlepage_da}
\aautitlepage{%
  \danishprojectinfo{
    Biograf Nordic % Title
  }{%
    Agil udvikling % Theme
  }{%
    Efterårssemestret 2020 % Project period
  }{%
    Gruppe 1 % Project group
  }{%
    % List of group members
    Daniel Thomsen\\ 
    Jesper Mellergaard\\
    Kasper Møller Nielsen\\
    Kristoffer Aagard Mikkelsen
  }{%
    % List of supervisors
    Henrik Kristian Ulrik Øllgaard
  }{%
    1 % Number of printed copies
  }{%
    21. december 2020 % Date of completion
  }%
}{% Department and address
  \textbf{Datamatiker}\\
  Professionshøjskolen UCN\\
  Sofiendalsvej 60\\
  \href{http://www.ucn.dk}{http://www.ucn.dk}
}{% The Abstract
%The Goal:
This team consisting of four computer science students will set out to expand our knowledge 
and experience concerning webservices, applications and software development processes. 
Our aim - to make an online cinema platform and an additional administration platform.
Various techniques and implementations learned in the academic year will be used to accomplish the goal 
- some explored by the team on its own. \\
%The methods:
The team has chosen to use an agile software development process; 
Scrum framework and extract some practises from Extreme programming. \\
%The result
The application works to the extend of the user story “book movie” and consists of a backend written in C\#, 
connected to an MSSQL database, an administration panel with simple functions using the 
backend API, and VueJs, HTML5, JavaScript and CSS frontend website. \\
%The conclusion:
The learning curve of the team has been great on our own accords and with more 
time would have build the application to its final state. 
Naturally the learning of new languages, implementations and techniques have slowed the process down 
but given this the group is satisfied with the product.
}}