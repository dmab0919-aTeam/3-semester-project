\section{Vandfaldsmetoden}\label{sec:vandfald}
Vandfalds-modellen er en sekventiel softwareudviklingsmetode som består af 5-7 forskellige faser. 
Navnet vandfald er blevet brugt, da det ikke er muligt at vende tilbage til foregående faser, 
når først en fase er færdiggjort. Resultatet fra hver fase er et krav for at kunne starte på den næste. 
Hvis kunden kommer med store ændringer til kravene, bliver hele processen nød til at starte forfra.  
Vandfalds-modellen kan ses på Figur ?? \\

Billede her. \\

Denne metode kan virke godt I forbindelse med mindre projekter, hvor det afsluttende produkt er 
helt defineret og det ikke forventes at ændringer kan forekomme. Bliver projektet stort og kompliceret,
hvilket ofte medfører, at der på sigt vil være ændringer i enten arkitekturen eller selve funktionaliteten, 
hvilket denne model ikke er udarbejdet til at tage højde for.

\section{Agile metoder}\label{sec:agilemetoder}
Agil er en betegnelse for en række iterative softwareudviklingsmetoder, hvor der løbende leveres små dele af 
produktet til kunden. Agile udviklingsmetoder er oftest kendetegnet ved Det Agile Manifest KIDLE!!!.
Manifestet beskriver bl.a. hvordan samtaler og møder med kunden, samt god, fungerende software er i fokus, 
fremfor en lang liste af processer og værktøjer og fuldstændig dokumentation.

\section{Metodevalg}\label{sec:valgafprocesmodel}
Hvilken metode er bedst til dette projekt? Det er et besværligt spørgsmål når der findes mange processer 
og metoder til softwareudvikling og produktudvikling generelt. Hvilket værktøj er bedst: en boremaskine 
eller en skruetrækker? Afhængigt af hvem man spørger kan svaret varierer og hvad der er det “rigtige” svar, 
kommer an på hvilket resultat man forventer at få af de værktøjer man anvender. Nogle gange kan det være nødvendigt at anvende begge. \\

Agile udviklingsprocesser minder om denne metafor, hvor vi i dette projekt har valgt at anvende Scrum, 
fordi undervisningen har introduceret os herfor og blandet det med nogle praktikker fra XP, grundet den værdi gruppen mener de tilfører til udviklingen.