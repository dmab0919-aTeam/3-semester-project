For at sikre at systemet bliver implementeret på den bedste måde for kunden, 
er det vigtigt at kigge på hvordan systemet skal bygges op. 
Arkitekturen ifølge Scrum \cite{ScrumArchitecture}, er ikke noget man bestemmer inden processen går i gang,
men derimod noget som opstår, løbende ud fra hvilke patterns osv. der bliver benyttet. 
Teamet bestemmer selv hvordan de forskellige Backlog-items skal implementeres, 
så længe de opfylder ”Definition of Done”. Arkitekturen for systemet bliver derfor løbende bygget op, 
og er derfor et resultat af ”Definition of Done”. \\

Dette er i stor kontrast til UP som gruppen arbejdede med på 2. semester. 
Her var det vigtigt at have styr på hvordan systemet skulle implementeres, 
i form af use-cases, klasse- og kommunikationsdiagrammer, og diverse andre modeller, 
inden implementeringen kunne begynde. 