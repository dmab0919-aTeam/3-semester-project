\section{Collective Code Ownership}\label{sec:collective}
Der er to principper indenfor ejerskab, enten individuelt ejerskab eller fælles ejerskab. 
Individuelt ejerskab er at den enkelte udvikler tager og har ansvar for sin egen kode. 
Et fælles ejerskab er, at have en fælles forståelse for projektet og alle har og tager ansvar for alt kode. 
Alle arbejder sammen om et fælles mål og alle er på lige fod med hinanden. 
Det kan være med til at sikre deling af viden, en bedre læringskurve, kode kvalitet og bus-effekten.
Bus-effekten betyder, at hvis én person forlader teamet, vil de andre have svært ved at forstå noget af koden, 
da et tidligere gruppemedlem stod for dette.