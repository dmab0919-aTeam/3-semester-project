\section{FURPS+}\label{sec:furps}
FURPS+ er et begreb der er hyppigt brugt inden for identificering af ikke-funktionelle krav. Nedenfor vil elementerne i FURPS+ 
blive gennemgået for dette projekt. \\

\textbf{Functionality:} \\
Functionality beskriver hvad det er kunden gerne vil have, altså hvilket produkt. Functionality tager ikke højde for sikkerhed, udseende og den slags.
\begin{itemize}
    \item Det endelige produkt skal være en platform, som kan håndtere booking af biograffilm.
    \item Kunden skal registrere sig ved køb.
    \item Kunden skal have mulighed for at forudbestille snaks til en film.
    \item Kunden skal kunne se forskelligt data på din profil side.
    \item Kunden skal føle at systemet er sikkert at benytte \\
\end{itemize}

\textbf{Usability:} \\
Usability er hvor nemt eller kompliceret produktet er at anvende. 
\begin{itemize}
    \item Produktet vil blive anvendt af den del af befolkningen som elsker at se film.
    \item Kunderne vil anvende det på deres telefon, computer eller tablet.
    \item Kunderne skal have mulighed for, at få hjælp gennem en chatfunktion.
    \item Det er vigtigt at systemet er nemt at bruge og intuitivt.
\end{itemize} 

\textbf{Reliability:} \\
Hvor pålideligt er systemet, altså kan kunden regne med at systemet ikke går ned og hvad sker der hvis det gør.

\begin{itemize}
    \item Systemet/hjemmesiden skal være tilgængelig hele året rundt på en adresse.
    \item Det forventes at en kopi af systemets data ligger på en backup server, hvor der laves backups flere gange om dagen, 
    i tilfælde af system nedbrud.
\end{itemize}

\textbf{Performance:} \\
Hvor hurtigt reagerer systemet, hvor langsom responstid er der og hvor meget data kan det sende eller hvor mange kunder kan anvende det samtid. Hvor meget hukommelse anvender det.
\begin{itemize}
    \item Det er vigtigt at systemet er hurtigt, da det forventes brugerne/kunderne, er utålmodige.
    \item Der skal være så få klik som muligt for at oprette en booking.
    \item Der skal være sigende fejlbeskeder, hvis noget går galt. \\
\end{itemize}

\textbf{Supportability:} \\
Kan det testes, udvides, vedligeholdes, installeres eller konfigureres nemt?
\begin{itemize}
    \item Systemet skal være testbart uden at gå på kompromis med sikkerheden.
\end{itemize}