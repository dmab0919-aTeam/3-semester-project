\subsection{Sprint 0 - Initial setup}
Dette projekts Sprint 0 begyndte d. 28/10 og sluttede 2 uger senere d. 11/11. Forud for dette sprint havde
gruppen programmeret en smule API og front end, uden rigtig at benytte et værktøj til Scrum. Sprint 0
blev derfor brugt på at færdiggøre det tekniske setup med hensyn til databasen og systemets overordnede struktur.
Nedenfor ses retrospektiv for Sprint 0, lavet d. 11/11: \\

\textbf{Start:} 28/10-2020 \\
\textbf{Slut:} 11/11-2020 \\

\textbf{Hvad gik godt?}
\begin{itemize}
    \item \textit{Pair programming:} Gruppen benyttede pair programming til de fleste tasks i Sprint 0. Dette resulterede
    i en færre kodefejl, og generelt set en mere flydende proces, da hvert gruppemedlem har forskellige styrker og svagheder.
    \item \textit{Daily standup meetings:} Gruppen havde prositive erfaringer med at holde et kort møde hver morgen, 
    med fokus på opsummering for at skabe et overblik over alle gruppemedlemmers nuværende opgave, og om tidsplanen holder.
\end{itemize}

\textbf{Hvad kunne have været gjort bedre?}
\begin{itemize}
    \item \textit{Et fast mødested:} Gruppen har indtil nu regelmæssigt skiftet lokation i forhold til hvor der er blevet lavet projektarbejde fra dag til dag, 
    hvilket har resulteret i, at gruppen hver aften har aftalt mødested til dagen efter. Dette har i nogle tilfælde skabt mindre misforståelser.
    \item \textit{Bedre brug af IceScrum:} Vi har i sprint 0 brugt en del af vores effektive tid på at finde ud af hvordan IceScrum fungerer
    i forhold til sammenhængen mellem backlog/planning/sprint/burndown.
\end{itemize}

\textbf{Hvad vil gruppen fokusere på at gøre bedre i næste sprint?}
\begin{itemize}
    \item \textit{Fast mødested og tid:} Gruppen vil fremover aftale et fast sted at mødes hver dag, der ikke er undervisning.
    \item \textit{Estimering af user stories:} Indtil videre har gruppen ikke gjort særlig ud af at estimere user stories/tasks på en formel måde.
    \item \textit{Code reviews:} Gruppen vil forsøge at være bedre til at gennemgå hinandens kode når den er færdig, i stedet for at vente til sidst
    i projektperioden.
    \item \textit{Løbende dokumentation:} Gruppen vil forsøge at dokumentere arbejdet mere løbende, eventuelt i forbindelse med code reviews.
    \item \textit{IceScrum:} Gruppen har i sprint 0 stiftet bedre bekendelse med værktøjet IceScrum, og vil i næste sprint benytte værktøjet efter bedste evne.
\end{itemize}