\subsection{Sprint 1}
Projektets sprint 1 startede d. 12. november, og sluttede to uger senere d. 25. november.
Formålet med dette sprint var at komme godt igang med projektet, og samtidigt følge principperne i Scrum. \\
Nedenfor ses retorspektivet fra sprint 1, lavet d. 25/11: \\

\textbf{Start:} 12/11-2020 \\
\textbf{Slut:} 25/11-2020 \\

\textbf{Hvad gik godt?}
\begin{itemize}
    \item \textit{Fast mødested og tid:} Gruppen aftalte efter sprint 0, at have et fast sted at mødes hver dag, for at undgå 
    forvirring i forbindelse med dag til dag planlægning. Dette er blevet overholdt efter hensigten, og har skabt færre misforståelser. 
    \item \textit{Dokumentation til rapport:} Gruppen har i sprint 1 været bedre til at skrive rapport løbende i projektarbejdet, 
    i modsætning til sprint 0, hvor der blev skrevet en meget begrænset mængde rapport. 
\end{itemize}

\textbf{Hvad kunne have været gjort bedre?}
\begin{itemize}
    \item \textit{Opdatering af scrum værktøj:} Gruppen oplevede udfordringer med IceScrum, uddybende dets funktionaliteter 
    og opnåede ikke en forståelse heraf, men hertil fandt et for gruppen, mere sigende værktøj.  
    \item \textit{Estimering af user stories:} Gruppen fik aldrig rigtigt estimeret user stories/tasks i sprint 1. Dette var delvist pga. 
    for få, og for store user stories. Dette resulterede i at gruppen på intet tidspunkt var helt klar over hvad der skulle nås i den to uger lange periode. 
    \item \textit{Code reviews:} Gruppen aftalte efter sprint 0, at forsøge på at blive bedre til løbende at lave code reviews, idéelt efter færdiggørelsen
    af hver task. Dette er ikke lykkedes efter hensigten.
\end{itemize}

\textbf{Hvad vil gruppen fokusere på at gøre bedre i næste sprint?}
\begin{itemize}
    \item \textit{Bedre brug af scrum:} Gruppen har valgt at skifte værktøj fra IceScrum til YouTrack. Dette skyldes at gruppen havde problemer med IceScrum, hvilket
    havde indflydelse på motivationen til at benytte værktøjet optimalt. Gruppen vil i næste sprint fokusere på at benytte Scrums elementer optimalt, og dermed være i stand
    til at præsentere et burndown-chart til næste sprint review.
    \item \textit{Code reviews:} Gruppen vil igen i næste sprint forsøge at lave code reviews. Hypotesen er, at dette vil blive nemmere, hvis gruppen benytter sig ordentligt 
    af et scrum board, da det ny vil blive tydeligt, hvad der mangler reviews.
    \item \textit{Fokus på produktudvilking:} Gruppen kom et godt stykke vej med rappportskrivning i sidste sprint, og vil i sprint 2, fokusere lidt mere på at udvikle produktet.
\end{itemize}