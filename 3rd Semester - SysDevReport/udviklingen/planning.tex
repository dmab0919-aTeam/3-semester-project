\section{Planlægning og estimering}\label{sec:planning}
Som nævnt i Afsnit \ref{sec:scrum} er en af Scrum's principper, ceremonien \textit{sprint planning}.
Product Owner’en har sammen med kunden og stakeholders udvalgt stories som skal prioriteres i næste sprint (altid flere end dem der kan være i sprintet). 

Product Owner’en og udviklerne vælger de stories som er mest hensigtsmæssige, ud fra dem der er prioriteret.
Hvis de 30 stories som er prioriteret alle har samme estimat, belyses de nøvendige teknologier og udfordringer. 
De stories som skal medtages i næste sprint vælges efter størst sammenhæng.\\

Scrum Masteren blander sig ikke i processen om udvælgelse af stories. Derimod forsøger Masteren at stille spørgsmåltegn ved eventuelle tekniske 
udfordringer, for at sikre dækningsgraden af overvejelserne.\\

\textbf{Gruppens Planning}
Gruppen udregnede sin \textit{velocity}, altså mængden af mandetimer til rådighed inden hvert sprint.
Estimeringen af de udvalgte tasks fra user stories'ne, skete ved hjælp af \textit{planning poker}. Planning poker er blevet udført på følgende måde:

\begin{itemize}
    \item Medlemmerne i scrum teamet giver hver især et bud på hvor langt tid en opgave vil tage i timer.
    \item Er der uenighed, snakker gruppen sammen om opgaven og når til en fælles enighed.
    \item Denne proces fortsætter indtil at den samlede estimerede tid på opgaverne, er den samme som gruppens planlagte velocity.
\end{itemize}

Projektgruppen planlagde hvert sprint på denne måde, men grundet manglende erfaring gik det ikke altid helt efter planen, blandt andet på grund
af fejlestimationer. Dette fremgår af gruppens retrospektiver, som ses i næste afsnit.