\section{Planlægning og estimering}\label{sec:planning}
Som nævnt i Afsnit \ref{sec:scrum} er en af Scrum's principper, ceremonien \textit{sprint planning}.
Sprint planning går ud på, at udviklingstemaet, sammen med product owner'en og scrum master'en, går igennem
backloggen og udvælger de stories/tasks som skal laves som det næste. Forud for dette, har gruppen udregnet sin 
\textit{velocity}, altså mængden af mandetimer til rådighed i næste sprint. \\

Nu skal gruppen vælge de tasks, der skal indgå i sprintet. Dette kan foregå på mange forskellige måder.
I dette projekt, har gruppen benyttet sig af \textit{planning poker}. Planning poker er blevet udført på følgende måde:

\begin{itemize}
    \item Scrum masteren vælger den højst prioriteret task fra backlog'en.
    \item Medlemmerne i scrum teamet giver hver især et bud på hvor langt tid en opgave vil tage i timer.
    \item Er der uenighed, snakker gruppen sammen om opgaven og når til en fælles enighed.
    \item Denne proces fortsætter indtil at den samlede estimerede tid på opgaverne, er den samme som gruppens planlagte velocity.
\end{itemize}

Projektgruppen har planlagt hvert sprint på denne måde, men grundet manglende erfaring er det ikke altid gået helt efter planen, blandt andet på grund
af fejlestimationer. Dette vil fremgå af gruppens retrospektiver, som ses i næste afsnit.