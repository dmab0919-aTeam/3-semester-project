\subsection{Sprint 3}
Projektets sprint 3 begyndte d. 3. december, og sluttede en uge senere, d. 9. december.
Målet for dette sprint, var at teste systemet, og have et øget fokus på Scrum. Forud for dette sprint
har gruppen i mindre grad forsømt nogle af principperne i Scrum. 
Nedenfor ses retrospektivet for Sprint 0, lavet d. 9/12: \\

\textbf{Start:} 3/12-2020 \\
\textbf{Slut:} 9/12-2020 \\

\textbf{Hvad gik godt?}
\begin{itemize}
    \item \textit{Testing:} Gruppen havde i sprint 3 fokus på at teste systemet gennem integrationstests. Gruppen har i denne uge fået 
    lavet en tilstrækkelig mængde tests for de mest centrale funktioner i systemet.
    \item \textit{Brug af Scrum:} Gruppen har igen i dette sprint forbedret sin evne til at benytte scrum efter hensigten, og opdatere 
    det valgte scrum-værktøj hver dag, for at producere et præcist burndown-chart.
    Gruppen har uploaded koden til en server, og derved fået bekræftet at det virker i praksis, og at arkitekturen virker efter hensigten. 
    Gruppen har derved en decentraliseret applikation.
\end{itemize}

\textbf{Hvad kunne have været gjort bedre?}
\begin{itemize}
    \item Gruppen kunne have færdiggjort user storien “create booking”.
    \item \textit{Estimering af tasks:} Gruppen har haft fejlestimeret diverse tasks, hvilket har resulteret i en urealistisk tidsplan, som har været
    svær at overholde.
\end{itemize}

\textbf{Hvad vil gruppen fokusere på at gøre bedre i næste sprint?}
\begin{itemize}
    \item \textit{Færdiggørelse af kode:} Gruppens mål for næste sprint er at gøre kodebasen færdig.
    \item \textit{Dokumentation:} I takt med at programmet bliver færdig, vil gruppen fremover have mere fokus på dokumentation af arbejdet.
\end{itemize}