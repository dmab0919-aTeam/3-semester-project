\chapter{Refleksion}\label{ch:refleksion}
Dette semester, på lige fod med sidste semester projekt, blev påvirket af de udfordringer covid-19 har medbragt.
Både i forhold til delvis online undervisning og delvis normal undervisning, 
men også i forhold til hvor meget gruppen har arbejdet i samme rum. \\

Det siges at vores generation er vant til at ”leve” online, men gruppen er enige om, 
at menneskelig kontakt og personlig kommunikation er svært at undvære for at sikre et optimalt forløb. 
Dette både i forhold til beslutningstagning, diskussioner og pair programmering og 
nemmere at kunne trække på hinandens færdigheder. \\

Anvendelsen af den nye agile udviklings framework Scrum og nogle Extreem Programming practices, 
har været et godt værktøj for gruppen i forhold til UP. \\

Et helt år er gået i selskab med covid-19 og endnu et semesterprojekt afsluttet. 
Gruppen har nydt at arbejde internt og alle medlemmerne har lært en masse nyt 
omkring software, processer, teknologier, agile udviklingsmetoder mm. \\

Produktiviteten, kommunikationen og produktet er helt tydeligt blevet bedre end sidste 
projekt og gruppen er endnu engang blevet dygtigere fagligt, 
men også lært hinandens styrker og svagheder endnu bedre. Resultatet heraf er, 
at vi hurtigere kan identificere hvem der er bedst egnet til, 
at løse en udfordring og hvem der har brug for støtte på et bestemt område. \\

Produktet er blevet godt, men der er altid plads til forbedringer, 
hvilket kommer af den læring gruppemedlemmerne har opnået af at fejle, 
ændre strategi og ikke give op på de udfordringer vi har mødt.
