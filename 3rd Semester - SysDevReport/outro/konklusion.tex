\chapter{Konklusion}\label{ch:konklusion}
For at konkludere på system udviklings perspektivet af projektet, refereres der til projekets problemformulering,
som blev udformet ved semesterets start.

\begin{center}
\textit{\textbf{Problem statement} \\
"How can the team design and implement a new universal booking system dedicated to movie theateres, 
that solves modern booking problems within this demanding industry?"} \\

\textit{\textbf{Subquestions}\\
"How can the team through a system development method optimize the proccess and mimimize errors along the way?"}\\

\textit{"How can a software development approach improve the quality and 
produced work of the development team throughout this project?"}\\
\end{center} 

%Hovedepunkter og resultater
Det konkluderes at gruppen har arbejdet udfra en agil udviklingsmetode, hvilket har været en kombination af Scrum og XP.
Taget i betragning af at gruppen netop på dette semester er blevet eksponeret for disse nye metoder og tankeprocesser,
vil kvaliteten afspejles heraf. \\

%Svar på problemformuleringen
Gruppen har kunnet opretholde en vis kvalitet af systemet, ved brugen af de redskaber som Scrum og XP tilbyder.
Daily standup meetings har sikret transparanthed i alle medlemmers opgaver og udfordringer, hvilket har sikret
en minimering af fejl i forhold til sidste semesterprojekt. 
Retrospectives har hjulpet gruppen med at finde diverse udfordringer, og sikret forbedring heraf. Dette har medvirket
til en optimering af gruppens arbejde, og sørget for udviklingen af processen. \\

%Vurdering af scrum
Det konkluderes i sammenhæng med overstående, at Scrum og XP kan forbedre kvalititen af det samlede arbejde, 
som vi som gruppe har udarbejdet i forbindelse med dette projekt. Problemformuleringens underspørgsmål anses derfor
som værende besvaret