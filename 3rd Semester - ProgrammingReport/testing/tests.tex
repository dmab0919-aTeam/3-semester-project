\section{Test Eksempler}\label{sec:testeksempler}
Nedenfor findes udvalgte sektioner af koden til demonstration af gruppens implementeringen. Den første kode sektion vist nedenfor er en
setup metode, som i dette tilfælde bliver kørt før hver test i den testklasse. Klassen hedder "ShowingControllerTests" og indeholder to metoder.
Hernæst vises en kode som tester et kald til API'et, hvor det kræver en "Authentication token" for at få tilladelse.\\

\textbf{Setup Before Test}\\
Først sættes email og password til at matche en af administratorene i systemet. Hernæst instansieres en "jsonbody" og parametrene indsættes.
Et "Request" oprettes og pejer på ../auth/login som er ruten til "AuthControlleren's" login function. 
"jsonbody'en" bliver tilføjet til "Requestet" og et IRestResponse modtages som respons på linje 20. 
Hernæst opsamles den "JWT", som skal anvendes til de andre test-metoder. 
Det gøres ved at deserialisere objektet, for at opsamle dens "Token" som ses på linje 27.

\lstinputlisting[language={[Sharp]C}, caption={Init()}, label={list:testsetup}]{Code/testadminloginsetup.txt}

\textbf{Metode Eksemple}\\
På linje 6 til 9, instansieres et "GET" request som kaldes på API'et med en route til ShowingControllerens "GetAllShowings" metode. Hernæst tilføjes
en header som indeholder den token som blev oprettet ved login'et (Listing \ref*{list:testsetup}, linje 27).\\

På linje 11 til 14, eksekveres requestet og herefter deserialiseres til en liste af showings. Linje 16 og 17 tester henholdsvis at status koden
er korrekt og at listen ikke er tom.\\

\textbf{Åbenbaring}\\
Det er vigtigt at tilføje negative tests, altså tests hvor resultatet forventes at komme tilbage negativt. Et eksempel på dette, er 
at ovenstående test bliver udført, men ved at logge ind som en almindelig bruger først. Her ville resultatet skulle forventes at være
negativt, idet den token en bruger og administrator token ikke er ens. Brugeren har derfor ikke rettigheder til at foretage kaldet.

\lstinputlisting[language={[Sharp]C}, caption={GetAllShowings(HttpStatusCode expectedHttpStatusCode)}, label={list:testgetallshowings}]{Code/testingadminloginmetode.txt}