\section{Authentication}\label{sec:auth}
Som mange andre it systemer, har biograf system også behov et login system til dets brugere, og administratorer.
Dette er blevet implementeret i Api'et, hvor man kan kalde et login endpoint med en email og password. 
Systemet vil herefter undersøge om der findes en bruger i databasen som matcher de indtastede oplysninger.
Dog bliver der fortaget flere sikkerhedsforanstaltninger, såsom hasing af passwordet som ligger i databasen,
for at undgå det står i plaintekst. Dette kan være praktisk i tilfælde af data breach.\\

\subsection{Password hashing \& Salting}
For at forbygge mod evt misbrug af database leak bliver passwordsne hashet med SHA256 algoritmen for at undgå passwordsne
bliver fremvist i plain tekst i tilfælde af database leak. Dette bliver gjort ved bruger oprettelse hvor brugerens indtastede 
password bliver krypteret med SHA256 algoritmen, og derefter gemt i databasen. Derfor skal man også hashe det indtastede 
passwordet når man logger ind i systemet, eftersom det ikke er muligt at sammenligne et krypteret password, og et ikke krypteret password.
I tilfælde af database leak, og en 'hacker' potentielt kan stjæle dataen, kan det være muligt for vedkommende at indtastede
de hashede passwords i et rainbow table, som er et arkiv for hashed passwords hvor man kan se den ukrypteret string af hashen.
For at undgå dette anvender systemet salt, som er en random string der bliver lavet med et 'Globally Unique Identifier' (guid).
som er et unikt id, der bliver tilføjet til password strengen. Dermed er hashen beskyttet mod rainbow tabel, eftersom hashen 
bliver unik, efter man tilføjer guid til passwordet. Salten bliver også gemt i databasen, så den kan anvendes til hasning
når brugern logger ind i systemet.\\

\subsection{Autrorize \& Roles}
I forbindelse med udvikling af api'et, er der forskellige endpoints som har varierende formål. Fx. Vil det være
uhensigtsmæssigt at alle kan kalde endpoints som er dedikeret til administrationspanelet, hvor det blot burde være
en administrator som kan kalde disse endpoints. Det samme er gældende for de endpoints som omhandler betaling og
resevation bestilling af biograf sæder. For at løse denne mindre problemstilling, og undgå evt misbrug af disse 
endpoints, er der blevet implementeret et authentication system som har til formål at beskytte disse endpoints.

\lstinputlisting[language={[Sharp]C}, caption={Endpoint middleware}, label={list:authorize}]{Code/auth.txt}

Som det fremgår i listing \ref{list:authorize} bliver disse endpoints beskyttet med [Authorize] som betyder at
endpointet kun kan blive kaldt hvis man er logget ind i systemet. Udover authorize af endpointsne er de også beskyttet 
med forskellige typer af roller, Så en user som er logget ind, ikke kan tilgå administrations endpoints.
Som det fremgår i følgende eksempel: "[Authorize(Roles = "Admin")]" vil dette endpoint kun være tilgængelig
for en administrator. Herved opnår vi beskyttelse af de de endpoints som ikke må tilgåes uden authentication.