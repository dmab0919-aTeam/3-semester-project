\section{Authentication}\label{sec:auth}
Som mange andre it systemer, har biograf systemet behov for et login-system til dets brugere, og administratorer.
Dette er implementeret i API'et, med et login endpoint som kræver email og password. 
Systemet undersøger om en bruger i databasen matcher de indtastede oplysninger.
Sikkerhedsforanstaltningerne, som anvendes i denne forbindelse er blandt andet, 
hashing af passwordet som ligger i databasen. Dette sikre at passwordet ikke læsbart.\\

\subsection{Password hashing \& Salting}
For at forbygge mod evt misbrug af database-leaks bliver et password hashet med SHA256 algoritmen for at undgå fremvisning 
i plain tekst. Ved registrering af en ny bruger vil det indtastede 
password blive indsat som parameter i metoden HASHPASSWORD i "Encrypt" klassen. 
Metoden tager et "Salt" som genereres for hver ny bruger og sammensættes med det password brugeren har valgt. 
til sidst bliver det hashet med SHA256 algoritmen, og derefter gemt i databasen. 
Det indtastede passwordet skal hashes når en bruger logger ind i systemet, eftersom det ikke er muligt at sammenligne et krypteret password, og et ikke krypteret password.
I tilfælde af database-leak, hvor en 'hacker' potentielt kan stjæle dataen, kan det være muligt for vedkommende at indtastede
de hashede passwords i et rainbow table, som er et arkiv for hashed passwords.
For at undgå dette anvender systemet salt, som er en random string der bliver lavet med et 'Globally Unique Identifier' (guid).
som er et unikt id, der bliver tilføjet til password strengen. Dermed er hashen beskyttet mod rainbow tabel, eftersom hashen 
bliver unik, efter man tilføjer guid til passwordet. Salten bliver gemt i databasen, så den kan anvendes til hasning
når brugeren logger ind i systemet.\\

\subsection{Autrorize \& Roles}
I forbindelse med udvikling af API'et, er der forskellige endpoints som har varierende formål. Det ville være
uhensigtsmæssigt at alle kan kalde endpoints som er dedikeret til administrationspanelet, og tilegnet 
en administrator. Det samme er gældende for de endpoints som omhandler betaling og
reservation af biograf sæder. For at løse udfordringen, og undgå evt misbrug af disse 
endpoints, er et authentication system implementeret til beskyttelse af disse endpoints.

\lstinputlisting[language={[Sharp]C}, caption={Endpoint middleware}, label={list:authorize}]{Code/auth.txt}

Som det fremgår i listing \ref{list:authorize} bliver endpoints beskyttet med [Authorize] som betyder at
endpointet kun kan blive kaldt hvis man er logget ind i systemet. Endpointsne er beskyttet 
med et rolle system, så en user som er logget ind, ikke kan tilgå administrations endpoints.
Som det fremgår i følgende eksempel: "[Authorize(Roles = "Admin")]" vil dette endpoint kun være tilgængelig
for en administrator. Herved opnår vi beskyttelse af de de endpoints som ikke må tilgåes uden authentication.