\section{Input validering}
I forbindelse med udvikling af et REST API er det vigtigt at have validering af de forskellige inputs som fremgår i request body'en.
Hvis en bruger skal indtaste en email addresse, eller et telefon nummer, er det ikke godt nok,
kun at chekke for hvilken type, eksempelvis string eller int, inputtet er. 
I biograf systemet er der udviklet forskellige typer af input valideringer,
til diverse formål.\\

\lstinputlisting[language={[Sharp]C}, caption={Validation Rules.}, label={list:validation}]{Code/validate.txt}

Som det fremgår på eksempelet i listing \ref{list:validation} er der opsat validerings regler for et UserDTO objekt. 
Valideringsreglerne er beskrevet med Regular expression (Regex), som betyder det er muligt
at opstille krav til hvordan en string er konstrueret. \\

I email feldtet kræves det at strengen, 
skal indeholde et navn, og derefter et @ tegn, efterfulgt af et domænenavn. 
Hvis disse valideringer ikke bliver godkendt,
vil der blive retuneret en eller flere fejlbeskeder.
