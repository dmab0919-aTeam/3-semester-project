\section{Input validering}
I forbindlse med udvikling af et RestApi er det vigtigt at have validering af de forskellige inputs som fremgår i request body'en.
Dette kunne fx. Være hvis en bruger skal indtaste en email addresse, eller et telefon nummer. Hvor det ikke er godt nok,
kun at chekke for hvilken type inputtet er. I biograf systemet er der udvilket forskellige typer af input valideringer,
til forskellige formål såsom email.\\

\lstinputlisting[language={[Sharp]C}, caption={Validation Rules.}, label={list:validation}]{Code/validate.txt}

Som det fremgår på eksempelet i listing \ref{list:validation} er der opsat validerings regler for det forskellige 
feldter på UserDto objektet. Validerings reglerne er beskrevet med Regular expression (Regex), som betyder det er muligt
at opstille krav til hvordan en string er konstrueret. I email feldtet er det derfor påkrevet at stringen, 
skal indeholde et navn, og derefter et @ tegn efterfulgte et domainenavn. Hvis disse valideringer ikke bliver godkendt,
vil der blive retuneret en eller flere fejlbedskder der har til formål at indikere hvad der er forkert med de forskellige inputs.
