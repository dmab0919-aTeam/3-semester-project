\section{JSON Web Token (JWT)}
Eftersom login-systemet foregår over et API, kan det være svært for API serveren, at vide hvilke kald some er autoriseret
og hvilke som ikke er, eftersom API'et er stateless. Systemet anvender JSON Web Token (JWT)
som bliver generet i login endpointet og sendt med videre i responset til klienten hvis login forsøget er succesfuldt.\\

Efter login inkluderes denne JWT token i authorization header, som bliver sendt med i requestene fremover.
API'et validere JWT tokenen og tjekker om kaldet er autoriseret. JWT har en Time To Live (TTL) som i biograf systemet
er indstillet til 3 timer. Rammer TTL nul, vil brugeren skulle logge ind igen, og dermed få generet en ny token som erstatter den
gamle JWT i authorization headeren på klientens kald.\\ 

\subsection{Generate JWT}
Genering af JWT er en længere process, hvor man har et seed, som er en unik hashed string. Seeded fungere som en nøgle
til JWT tokens hvor man i teorien kan lave 'falske' JWT tokens hvis man kender til dette hemmelige seed. I Biograf systemet
bliver følgende string brugt som seed: "MynameisJamesBond007", dette bliver hermed hashed med HMAC-SHA256 algoritmen, 
som sørger for at JWT toknen bliver unik. Hertil bliver brugerens email og roller også indsat som et claim, som betyder
at det er information som kan uddrages fra JWT toknen, så API'et altid kan se hvilken bruger der prøver at lave et givent
request.\\