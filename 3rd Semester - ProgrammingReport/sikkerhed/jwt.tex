\section{JSON Web Token (JWT)}
Eftersom login systemet foregår over et api, kan det være svært for api serveren, at vide hvilke kald some er autoriseret
og hvilke kald som ikke er, eftersom api'et er stateless. Derfor anvender biograf systemet JSON Web Token (JWT)
som bliver generet i login endpointet og sendt med videre i respons svaret til klienten hvis login forsøget er succesfuldt.
Herefter vil klient fremover inkludere denne token i authorization header, som bliver sendt med i requestene fremover.
Herefter kan api'et validere JWT tokenen og se om kaldet er autoriseret. JWT har en Time To Live (TTL) som i biograf systemet
er indstillet til 3 timer. Herefter skal brugeren logge ind igen, og dermed få generet en ny token som så også erstatter den
gamle JWT i authorization headeren på klientens kald.\\ 

\subsection{generate JWT}
Genering af JWT er en længere process, hvor man har et seed, som er en unik hashed string. Seeded fungere som en nøgle
til JWT tokens hvor man i teorien kan lave 'falske' JWT tokens hvis man kender til dette hemmelige seed. I Biograf systemet
bliver følgende string brugt som seed: "MynameisJamesBond007", dette bliver hermed hashed med HMAC-SHA256 algoritmen, 
som sørger for at JWT toknen bliver unik. Hertil bliver brugerens email og roller også indsat som et claim, som betyder
at det er information som man kan få ud fra JWT toknen, så api'et altid kan se hvilken bruger der prøvet at lave et givent
request.\\