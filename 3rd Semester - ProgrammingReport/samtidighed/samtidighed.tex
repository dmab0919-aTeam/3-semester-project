\section{Samtidighed Ved Reservation}\label{sec:svr}
Der opstår udfordringer, hvis to eller flere kunder forsøger at reserver samme 
sæder til en film. Første udfordring er, at sæderne selvfølgelig kun kan bookes 
af en kunde. Anden udfordring er at opdatere den visuelle præsentation af 
reservationer på en ”Showing”. Heraf opstår dette spørgsmål:\\

\textit{Hvad sker der når to brugere prøver at reservere de 
samme sæder på samme tid?}\\

Uden forbehold ville resultatet være to kunder med billetter til same sæder og 
film, hvilket ville resultere i nogle forretningsmæssige negativer.\\

\textbf{Løsningen}\\
Første del af løsningen - At sikre tabellen mod duplikater. ”Seats” tabellen 
(se Bilag \ref{app:relationelmodel}) har en samlet primary key på Row, Number og ShowingID. 
Muligheden for at indsætte nye rækker hvor disse 3 kolonner er ens blev heraf 
fjernet.\\

Anden del af løsningen - I gruppens tilfælde at anvende et ”TransactionScope”. 
Hermed sikres atomicitet i transaktionen, alt eller intet princip, og sørger for 
at lave en ”Rollback” på transaktionen skulle den fejle.
Metoden ”AddSeatAsync()” (Ses i Listing \ref{list:addseatsasync}) tager en liste af de valgte sæder, itererer igennem 
listen og indsætter hver enkelt sæde i databasen. Brugeren får en fejl besked, 
hvis det forsøges at indsætte sæder som allerede eksisterer i databasen.

\lstinputlisting[language={[Sharp]C}, caption={AddSeatsAsync()}, label={list:addseatsasync}]{Code/addseatsasync.txt}

Tredje del af løsningen - At implementere en timer i frontenden, som opdaterer 
de tilgængelige sæder hvert 10. sekund ved at lave et ”Get Request” til api’et 
som henter de reserverede/købte sæder(Metoden i frontenden kan ses i Listing \ref{list:fetchseats}). 
På den måde vil brugeren ikke kunne forsøge at bestille sæder som ikke er 
tilgængelige.

\lstinputlisting[language={[Sharp]C}, caption={fetchReservedSeats()}, label={list:fetchseats}]{Code/fetchreservedseats.txt}

\section{Nye udfordringer - Part 1}\label{sec:part1}
En bruger reserverer et par sæder og undlader at betale således at ordren 
gennemføres. Tabellen ”Seats” indeholder en kolonne ”State” 
(se Bilag \ref{app:relationelmodel}), som anvendes til at identificere reserveret eller 
købte sæder. Brugerens valgte sæders ”State” ændres derfor til ”Reserved” i 
tabellen. Reservationen af sæder afvikles i forbindelse med at en bruger vælger 
sæder og trykker på knappen ”Continue”, og købet sker efter brugeren har logget 
ind og betalt.\\

\textit{Hvad sker der med reserverede sæder, som ikke bliver købt?}\\

\textbf{Løsningen}\\
Første del af løsningen - At tilføje en ny kolonne i Seats tabellen, ved navn 
”ReserveTime”. ”ReserveTime” er et tidsstempel på reservationen af valgte sæder. 
”ReserveTime” kan sammenlignes med row-version, som nævnt tidligere i kapitlet, 
er en del af optimistisk samtidighed.\\ 

Anden del af løsningen - At give brugeren ti minutter, efter reservationen 
af valgte sæder, til at bekræfte købet og gennemføre ordren.\\ 

Tredje del af løsningen - At anvende tidsstemplet ”ReserveTime” til at 
opdatere tabellen ”Seats” når en ny bruger tilgår sæde-selektionen, på samme 
”Showing”. Metoden ”DeleteOldSeatsAsync()” kaldes og fjerner reservationer 
ældre end 10 minutter fra databasen, ved at kigge på alle tidsstempler. 
Timeren i frontenden som opdaterer listen, kalder ligeledes denne metode.
Metoden kan ses i Listing \ref{list:deleteoldseats}.

\lstinputlisting[language={[Sharp]C}, caption={DeleteOldSeatsAsync()}, label={list:deleteoldseats}]{Code/deleteoldseatsasync.txt}

Querien i metoden ovenfor fjerner alle ”Seats” tilknyttet den bestemte ”Showing”, 
hvor State er ”Reserved” og som er ældre end ti minutter.\\  


\section{Nye udfordringer - Part 2}\label{sec:part2}
At fjerne reservationer ældre end ti minutter, er med til at skabe nogle 
nye problemstillinger, som gruppen har været nødsaget til at finde en løsning på.

\begin{enumerate}
    \item En bruger venter længere end ti minutter med at betale.
    \item En bruger venter længere end ti minutter med at betale, 
    hvor reservationen er blevet slettet, i det en anden bruger har 
    tilgået den samme showing.
    \item En bruger venter længere end ti minutter med at betale, 
    hvor reservationen er blevet slettet, i det en anden bruger har 
    tilgået den samme showing. Den anden bruger vælger de samme sæder 
    og reservere dem.
\end{enumerate}

\textbf{Første}\\
Løsningen til første udfordring - At en bruger skal være inde på 
reservations siden før listen, bliver opdateret hver tiende sekund. 
Det vil sige, at hvis en bruger reserverer nogle sæder og venter over ti 
minutter, men der samtid ikke er nogle nye brugere der besøger reservations 
siden, vil brugeren stadig kunne købe de valgte sæder. ”State” på de valgte 
sæder bliver opdateret til ”Bought” i databasen, og kan herefter ikke længere 
bestilles igennem sæde-selektionen.\\ 

\textbf{Anden}\\
Løsningen til anden udfordring - At et tjek udføres, om der findes en reservation på de valgte 
sæder, hvis ikke der gør det, får brugeren lov at købe. På den måde undgås det, 
at brugeren sendes tilbage til en reservations side, hvor de sæder brugeren 
forsøgte at købe stadig var ledige.\\

\textbf{Tredje}\\
Løsningen til tredje udfordring - At tilføje en ekstra kolonne på ”Seats” tabellen (se Bilag \ref{app:relationelmodel}), ved navn 
”UUID” (Universally unique identifier). Denne kolonne har til formål at holde 
styr på hvem der har reserveret sæderne. Ved brugen af en UUID-kolonne, 
bliver det altså ikke muligt at købe sæder, som andre har reserveret.\\  

\textbf{One method to rule them all}\\
Disse tre udfordringer, har gruppen formodet at inddæmme i én metode 
”BuySeatAsync()”, som bliver kaldt når brugeren trykker på ”Pay”. 
Querien i metoden, indeholder en ”If-Else” kæde. Den tjekker først om der 
findes nogle sæder i databasen som er reserveret, og som har brugerens UUID, 
på de valgte pladser. Hvis dette er tilfældet, bliver sædernes State ændret til 
”Bought”, og brugeren har derved købt sæderne.

\lstinputlisting[language={[Sharp]C}, caption={BuySeatsAsync() - forkortet}, label={list:buyseats}]{Code/buyseatsasync.txt}

Hvis dette ikke er tilfældet, tjekker querien om der findes nogle reserverede 
sæder på de valgte pladser, uden at lede efter UUID. Hvis der er reserverede 
sæder, printer querien en tom ”String”, hvilket returnere -1, som bliver 
tjekket i API ‘et, og brugeren får en fejl besked. Findes sæderne derimod ikke, 
indsætter den bare nogle nye sæder, som er ”Bought”. 

