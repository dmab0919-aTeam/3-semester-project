\textbf{Samtidighed / Concurrency}\\
Håndtering af multiple brugere i et system kræver, 
at udviklerne tager højde for eventuelle samtidighedsproblemer. 
I en situation, hvor to brugere ønsker at ændre eller tilføje de samme ressourcer, 
på samme tid, må kun én lykkedes. 
Der findes forskellige måder at løse disse problemstillinger på.\\

\textbf{Pessimistisk Samtidighed}\\
Ved pessimistisk samtidighed låses data som brugeren ønsker at ændre/håndtere. 
En bruger tilgår data i databasen og en tabel læses, her sættes der en lås på 
tabellen. Brugeren ændrer data i tabellen, hvilket resulterer i at andre brugere 
kun har læserettigheder på de låste tabeller. Tabellen låses op når brugeren er 
færdig med sine ændringer. Pessimistisk låsning bruges med fordel, hvis mængden 
af data er minimal og opdateres ofte eller indeholder følsomt data.\\ 

\textbf{Optimistisk Samtidighed}\\
Ved denne tilgang, er udgangspunktet at samtidighedsproblemer sjældent sker. 
Her låser man ikke tabellerne i databasen, men tilføjer en ekstra kolonne som 
fx et tidsstemple, dato eller row-version. Det fungerer lidt som først til 
mølle-princippet. Når brugeren går i gang med at opdatere en tabel, tjekker 
systemet den ekstra kolonne for at sikre det er den rigtige version. 
Hvis den samme tabel er blevet opdateret, inden brugeren har færdiggjort 
transaktionen, får brugeren en fejl, og må prøve igen.\cite{MSOC}
