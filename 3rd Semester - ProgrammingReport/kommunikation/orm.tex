\section{Object Relation Mapper (ORM)}\label{sec:orm}
En Object Relation Mapper, forkortet ORM, er en programmeringsteknik der bliver brugt til
at manipulere data fra en database i et objekt-orienteret program. ORM'en kan abstrakt set fungere som en bro mellem databasen, og de objekt orinterede klasser som kan bruges i forskellige sprog.   
\\

Dette projekt benytter sig af en SQL database. SQL databser er ikke objekt-orienteret, da de
blot kan gemme variabler som tal og strenge, organiseret i tabeller. Produktet til dette projekt
er skrevet i C\#, som netop er objekt-orienteret. Dette skaber grundlag for at benytte sig af en
ORM. 

\subsection{Valg af ORM}\label{dapper}
Projektgruppen har til dette projekt identificeret to mulige valg af ORM: Entity Framework og Dapper.
Gruppen blev præsenteret for Dapper i forbindelse med undervisningen i teknologi på 3. semester, og fandt Dapper
nemt at bruge i mindre projekter. Gruppen blev også præsenteret for Entity Framework, som skulle være mere omfattende
end Dapper, samt indeholder flere funktioner. \\ 
Gruppen har fundet frem til følgende fordele og ulemper ved henholdsvis brug af Entity Framework og Dapper: \\

% https://entityframework.net/ef-vs-dapper
\textbf{Entity Framework:}
\begin{itemize}
    \item Fordele:
    \begin{itemize}
        \item Entity Framework kan erstatte større kodeblokke, som programmøren ellers ville være nodt til at skrive og vedligeholde selv.
        \item Entity Framework kan autogenere kode.
        \item Entity Framework benytter én syntax til alle objekter (LINQ), om de er databaser eller ej.
    \end{itemize}
    \item Ulemper:
    \begin{itemize}
        \item Entity Framework har et højere entry-level end fx Dapper.
    \end{itemize}
\end{itemize}

\textbf{Dapper:}
\begin{itemize}
    \item Fordele:
    \begin{itemize}
        \item Dapper er meget "efficient" og har fået tilnavnet "King of Micro ORM" i forhold til performance.
        \item Dapper gør det nemt at parametrisere queries.
        \item Dapper gør det nemt at lave resultater om til objekter.
    \end{itemize}
    \item Ulemper:
    \begin{itemize}
        \item Dapper kan ikke generere en class model, og kan ikke selv generere queries.
        \item Dapper kan ikke tracke objekter og deres ændringer.
    \end{itemize}
\end{itemize}

Gruppen har på baggrund af ovenstående, vurderet at Dapper egner sig bedst til et mindre projekt som
dette, da Dapper giver mere kontrol over databasen, og fungerer bedre med "database first" end
Entity Framework gør. Endvidere, finder gruppen Dapper nemt at benytte i praksis, hvilket kan spare gruppen dyrebar
tid i projektperioden