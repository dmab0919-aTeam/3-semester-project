\section{Http}\label{sec:http}
https://www.cloudflare.com/learning/ddos/glossary/hypertext-transfer-protocol-http/
“Hypertext Transfer Protocol” eller HTTP er en tilstandsløs applikationslags protokol, 
som bliver anvendt til kommunikation mellem distribuerede systemer. 
Kommunikationen kan foregå mellem forskellige ”Hosts” og ”Clients”, 
og understøtter forskellige konfigurationer.\\

Protokollen formoder meget lidt om det bestemte system som kaldes, 
og holder ikke fast i dets tilstand mellem forskellige kald. 
Et ”REQUEST” bliver sendt af klienten, modtaget af hosten, 
et ”RESPONSE” sendt tilbage til klienten og herefter glemmer hosten klienten. \\

”Hypertext Transfer Protocol” er fundamentet for enhver webudvikler 
og en god forståelse heraf er essentiel. 
Det antages at langt de fleste brugere af internettet har fået en ”404 page” på et tidspunkt; 
En HTTP-status kode for ”Not Found”. \\

\textbf{Uniform Ressource Locators:}\\
Når man kigger på en hjemmeside-adresse, altså dens URL, 
er der en helt bestemt måde af aflæse den på. 
Tag dette eksempel: ”http://www.localhost.dk:80/customers”, 
her ses først ”http”, hvilket er protokollen som anvendes, på den specifikke hjemmeside. \\

Dette præfiks kunne også være ”https”, 
hvilket betyder at protokollen kører over en ”SSL-forbindelse”, 
altså en krypteret transmission. 
Det vil sige kun afsender og modtager kender til de meddelelser der sendes internt. \\

Hernæst ses værten ”www.localhost.dk”, porten ”80” og til sidst ressourcestien ”/customers”. 
Port 80 er standard port til HTTP, men kan ændres hvis det skulle være nødvendigt. \\

\textbf{Hypertext Transfer Protokol Verbs:} \\
”Hosten” som ønskes at kommunikere med, bliver identificeret gennem URL’en, 
men den handling som ønskes udført bliver specificeret gennem nogle verber. 
Dette system beskæftiger sig med disse: \\
\begin{enumerate}
    \item ”GET”: Hente en eksisterende ressource.
    \item ”POST”: Oprette eller skabe en ny ressource.
    \item ”PUT”: Opdatere en eksisterende ressource.
    \item ”DELETE”:  Fjerne eller slette en eksisterende ressource.
\end{enumerate}

I listing ?? (GetByID MoviesController) ses et eksempel fra koden, 
med en ”HttpGet” hvor ”requestet” kræver et ”id” i sin body. \\

I listing ?? (PostAsync ShowingController) ses et eksempel fra koden, med en ”HttpPost”, 
hvor ”requestet” forventer en body og ikke parametre i URL’en. 
Her ses også brugen af roller til autorisering og afsenderen (læs om det her).