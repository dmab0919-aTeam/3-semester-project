\section{API}\label{sec:api}
Et "application programming interface", eller API, er et interface der definerer interaktionerne
mellem flere programdele. Det definerer hvilke kald der kan blive foretaget, hvordan de skal laves, 
hvilke dataformater der skal benyttes, osv. API'er bruges i programmering til at abstrahere den underliggende
implementation ved kun at eksponere de objekter eller funktioner, programmøren har behov for. \\

\subsection{RESTful}
Projektgruppen er i undervisningen på 3. semester blevet præsenteret for RESTful API. RESTful er
baseret på "representational state transfer" REST, som er en fremgangsmåde der ofte benyttes i forbindelse med
udvikling af web services. Dette gøre Restful relevant til dette projekt, da en del af opgaven er web services.
RESTful bruger HTTP requests som GET, PUT, POST og Delete til at tilgå og bruge data. \\

REST er stateless, hvilket vil sige at serveren ikke gemmer nogle informationer om klienternes state på serversiden.
Enhver request fra klienterne skal derfor indeholde al information der er relevant for at forstå kaldet, og kan ikke
tage fordel i information der findes på serveren. 

\subsection{Brug af RESTful med RESTsharp}
% Kodeeksempler hvor vi bruger API'et skal præsenteres her