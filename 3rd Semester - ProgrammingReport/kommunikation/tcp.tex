\section{TCP / IP}\label{sec:tcp}
”Transmission Control Procotol” og ”Internet Protocol” \cite{TCP} er med til at gøre internettet muligt. 
Disse specificerer hvordan enheder deler data over internettet mellem sig. 
Protokollerne specificerer hvordan data skal brydes ned, håndteres, sendes, ruten og modtagelsen.\\

TCP definerer hvordan applikationer opretter kommunikationskanaler, 
og hvordan beskeder eller data skal brydes ned i mindre dele inden det sendes. 
De nedbrudte data segmenter sendes over internettet, 
og TCP bestemmer ligeledes hvordan det samles igen på destinationen. \\

IP (Internet Protocol) definere hvordan dataet skal adresseres, 
og hvilken rute det skal tage for at nå den rigtige destination; altså modtagerens destination. \\

Det skal forstås at TCP’en venter på at IP’en har oprettet den nødvendige forbindelse mellem to enheder, 
før den kan udføre sit arbejde. \\

\textbf{De fire lag i TCP/IP:} \\
Funktionaliteten er delt op i fire lag; Applikations-, transport-, netværks- og et datalinklag.\\

Applikationslag
\begin{itemize}
    \item Her findes standardiseret dataudvekslings protokoller. Bland andet ”HTTP” som gruppen anvender.\\
\end{itemize} 

Transportlag
\begin{itemize}
    \item Transportlaget sørger for en pålidelig forbindelse mellem to enheder. 
    Altså den vedligeholder end-to-end kommunikation på tværs af netværket med TCP-protokollen.\\
\end{itemize}

Netværkslag
\begin{itemize}
    \item Håndterer data pakker og forbinder netværker, 
    til at transportere pakkerne på tværs af netværks grænser, 
    ved brug af IP og ”Internet Control Message Protocol” ICMP. 
    Det håndterer dataets bevægelser fra afsender til modtager på internettet.\\
\end{itemize}

Datalinklag
\begin{itemize}
    \item Håndtere de fysiske dele ved at sende og modtage data, 
    ved brug af et ”Ethernet cable” eller trådløs netværks forbindelse osv.\\
\end{itemize}

