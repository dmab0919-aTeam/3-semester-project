\chapter{Konklusion}\label{ch:konklusion}
For at konkludere på system udviklings perspektivet af projektet, refereres der til projekets problemformulering,
som blev udformet ved semesterets start.

\begin{center}
\textit{\textbf{Problem statement} \\
"How can the team design and implement a new universal booking system dedicated to movie theateres, 
that solves modern booking problems within this demanding industry?"} \\

\textit{\textbf{Subquestions}\\
"How can the team through a system development method optimize the proccess and mimimize errors along the way?"}\\

\textit{"How can a software development approach improve the quality and 
produced work of the development team throughout this project?"}\\
\end{center} 

%Hovedepunkter og resultater
Det konkluderes at gruppen har udviklet og implementeret i bookingsystem til biografbilletter, der kan håndtere de 
mest centrale funktioner: At vælge en film, vælge et tidspunkt, loging, samt at vælge sæder, og derved generere en 
biografbillet. \\


%Svar på problemformuleringen
Gruppen har udviklet systemet, ved brug af teknologier der er blevet præsenteret i undervisningen på 3. semester. Herunder gælder teknologier
som Dapper som ORM, og Restful API ved brug af library'et RestSharp. Ifølge problemformuleringen, skal systemet kunne løse 
moderne problemer i forhold til booking på en hjemmeside. Her har gruppen identificeret to samtidighedsproblemer, som med success er blevet løst. \\


%Vurdering