\section{Desktop-Klient}\label{desktop}
Et af kravene til projektet, er en dedikeret desktop klient, som sammen med web-klienten, 
også skal kunne kommunikere med den udviklede service/API. 
I gruppens tilfælde, gav det mening at lave desktop-klienten til et administrator-panel, 
som medarbejdere ville kunne bruge, når data, i form af ”Movies/Showings”, skal opdateres på hjemmesiden. \\

Gruppen har valgt at benytte sig af en WPF (Windows Presentation Foundation) applikation. 
Applikationen indeholder kun et MainWindow. Koden kan ses i listing \ref{list:mainwindow}

\lstinputlisting[language={[Sharp]C}, caption={MainWindow - Forkortet}, label={list:mainwindow}]{Code/mainwindow.txt}

Det ses at MainWindow omdirigere medarbejderen til en bestemt IP-adresse, 
og sender personen til login siden. Det vil sige at administrator-panelet er lavet i Vue, 
ligesom frontenden til web-klienten. Eftersom koden er ”deployed” til en server, 
kræver det internet at kunne tilgå systemet. 
Hvis medarbejderen starter programmet uden internet, popper en fejl besked op, 
og medarbejderen bliver spurgt om at tjekke internetforbindelsen. \\

Når medarbejderen logger ind, får han/hun mulighed for at opdatere de tilgængelige ”Showings”, 
tilføje nye og slette dem. Det er også muligt at tilføje brugere og fjerne dem igen. 
