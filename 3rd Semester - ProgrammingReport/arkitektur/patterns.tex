\section{Design patterns}\label{designpatterns}
Softwareudvikling er i høj grad en holdsport. Ofte skal flere personer arbejde sammen om at udvikle et stykke software.
For at sikre at alle udviklere strukturerer deres kode ens, benyttes design patters. I dette projekt, er primært to design 
patterns blevet brugt: Repository Pattern, og Unit of Work Pattern. Disse vil blive beskrevet nedenfor.

\subsection{Repository Pattern}
I projektet NordicBio.dal ligger mapperne interfaces, repositories som bliver anvendt i forbindelse med udarbejdelsen af 
dette mønster, hvor et repository virker som et lager eller en samling af objekter for et specifikt domæne som formidler 
kommunikationen fra og til databasen. \\

\textbf{Hvorfor valgte gruppen dette pattern?}
Det er et mønster som gruppen er støt på, og kan se fordele ved at anvende, under undervisningen på 3. semester på uddannelsen. \\

Herudover er mønsteret testbart, da hvert ”lag” understøtter blot ét domæne. \\

I den forbindelse er der en lav kobling da hver ”lag” kun har et ansvar; at håndtere ét domæne og forbindelsen til databasen.

\subsection{Unit of Work Pattern}
Dette mønster bliver ofte anvendt i forbindelse med repository pattern hvilket også er tilfældet i dette projekt. \\

Dette mønster samler alle repositories i et objekt, hvilket gør det muligt at håndtere ændringer i flere domæner, 
hvis nødvendigt og pakke disse ændringer ind i et transactionscope altså en transaction som sikre princippet om ”alt eller intet”. 
Hvis der sker en fejl ved den første ændring eller sql statement vil hele transactionen lave en rollback. På den måde sikre systemet 
af hvis 2 eller flere afhængige kald ikke alle er successfulde vil ingen af dem blive gennemført.