\section{Design patterns}\label{designpatterns}
Udviklingen af store systemer foregår ofte i teams, nogle gange meget store teams. 
Når software udvikles kan det være en fordel at have fælles forståelse for hvilke mønstre som anvendes. 
På den måde kan det mindske risikoen for at udviklere laver kode som ikke matcher på struktur og tilgang. 
I dette projekt, er primært to design 
patterns blevet brugt: Repository Pattern, og Unit of Work Pattern. Disse vil blive beskrevet nedenfor.

\subsection{Repository Pattern}
I projektet NordicBio.dal ligger mapperne interfaces og repositories som bliver anvendt i forbindelse med udarbejdelsen af 
dette mønster, hvor et repository virker som et lager eller en samling af objekter for et specifikt domæne som formidler 
kommunikationen fra og til databasen. \\

Det kan uddrages af designklassediagrammet (diagrammet  beskrives i næste afsnit, 
og kan ses i Bilag \ref{app:designklassediagram}) at et 
"IGenericRepository" interface bliver anvendt til, 
at sikre de standard metoder alle repositories skal indeholde. 
Hernæst har hver repository sin egen interface med mere specifike metoder, hvis andre implementationer skal laves. \\

\textbf{Hvorfor valgte gruppen dette pattern?}\\
Det er et mønster som gruppen er støt på, og kan se fordele ved at anvende, under undervisningen på 3. semester på uddannelsen. \\

Herudover er mønsteret testbart, da hvert ”lag” understøtter blot ét domæne. \\

I den forbindelse er der en lav kobling da hvert ”lag” kun har et ansvar; at håndtere ét domæne og forbindelsen til databasen.

\subsection{Unit of Work Pattern}
Dette mønster bliver ofte anvendt i forbindelse med repository pattern hvilket også er tilfældet i dette projekt. \\

Dette mønster samler alle repositories i et objekt, hvilket gør det muligt at håndtere ændringer i flere domæner, 
hvis nødvendigt og pakke disse ændringer ind i et transactionscope altså en transaction som sikre princippet om ”alt eller intet”. 
Hvis der sker en fejl ved den første ændring eller sql statement vil hele transactionen lave en rollback. På den måde sikre systemet 
at hvis to eller flere afhængige kald ikke alle er successfulde vil ingen af dem blive gennemført. \\

Gennem dependency injection får hver controller en IUnitOfWork, 
som bliver deres adgang til alle repositories, hvilket på sin vis giver adgang til databasen.\\

UnitOfWork klassen implementere IUnitOfWork (se Bilag \ref{app:designklassediagram}), 
hvilket sikre den indeholder alle nødvendige Repositories.