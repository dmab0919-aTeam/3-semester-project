\section{Stack}
Dette afsnit har til formål at beskrive systemets opbygning samt beskrive hvilke teknologier som udgør systemet. 
Systemet er opbygget i en client-server arkitektur, hvor klienten har til formål at agere som applicationens grafiske 
brugerflade, som kan anvendes af systemets stackeholders. Serveren består et rest Api, som gør det muligt for klienten 
at kommunikere med databasen gennem serveren, og udføre beregninger som kan blive vist i klienten. Serveren er også 
ansvarlig for at oprette forbindelse til databasen. Derved er det kun muligt for klienten at tilgå data fra databasen 
ved at kommunikere gennem serveren.\\

\subsection{Server}
Serveren består af en DotNet Core Solution som indeholder flere projekter, hvor REST API’et befinder sig i NordicBio.api 
projektet med typen ”Microsoft.NET.Sdk.Web” som betyder det er beregnet til webapps. 
NordicBio.api indeholder api routes som kan blive kaldt af klienten, med hensigt på at tilgå data og gemme data fra databasen. 
Dette kunne fx være at modtage information om hvilke film som er tilgængelig i NordicBio, eller at sende information til 
NordicBio.api som skal gemmes i databasen.\\

\subsection{Klient}
Klienten er opbygget i Vue.js som er et open source JavaScript framework, der gør det nemmere at lave grafiske 
brugerflader til REST API’er. Eftersom Klienten er opbygget i Vue.js kan gui’en bygges i HTML og CSS som kan tilgås 
gennem en webbrowser. Klienten kan ved hjælp af POST og GET requests modtage og sende data fra serveren, 
gennem dens api routes.\\

\subsection{Sammenfatning}
I dette afsnit er der blevet redegjort for hvilke teknologier som bliver inddraget i systemet og dets opbygning. 
Ved at opdele serveren og klienten ind i flere services kan det være nemmere at vedligeholde eller udskifte teknologier 
i fremtiden, eftersom de forskellige services kommunikere gennem REST API’et. 
Dette betyder at Klienten ikke er afhængig af at REST API’et er udviklet i DotNet Core eller andre specifikke sprog så 
længe det stadig er et REST Api. Det samme gælder for serveren og databasen. Til gengæld kan det være mere uoverskueligt 
for mindre teams at understøtte flere teknologier i samme kode projekt. 


