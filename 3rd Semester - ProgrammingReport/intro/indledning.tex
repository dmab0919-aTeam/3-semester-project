\chapter{Indledning}\label{ch:indledning}
Den antropocæne tidsalder; ”Den nye menneskelige tidsalder”. Dette er den tidsalder mennesket befinder sig i.
En ny tid og hermed nye teknologier, tankeprocesser og et kampløb for mange virksomheder og mennesker. 
Et kampløb om at holde sig informeret, at ”være med” på nyeste mode, trend eller dille. \\

Dette gælder også i teamets valgte profession, hvor nye udviklingssprog, metoder og teknologier dukker 
hurtigere op på markedet end nogensinde før i softwarets korte levetid. 3. semester har eksponeret teamet for nye sprog og teknologier; 
C\#, HTML5 og CSS. \\

Teamet beskæftiger sig i denne opgave med at udvikle en web-løsning til en fiktiv problemstilling,
som omfatter biografbookingsystemer. Denne opgave formidles ikke i samarbejde med en virksomhed, 
men teamet agerer her som den fiktive kunde og varetager derfor kundens interesser, ønsker og prioriteringer. \\

Løsningen vil sammenfatte teamets kompetencer bygget på undervisningen fra 3. semester samt opsøgt viden løbende i 
projektet, hvilken sammenfatter teamets vidensgrundlag. \\

Rapporten vil omfatte al relevant dokumentation for projektets forløb; arkitekturen, 
kommunikation, samtidighed, sikkerhed og tests. Afslutningsvis en refleksion og konklusion som samler hele projektet op.

\newpage
\section{Problemformulering}\label{sec:problemformulering}
Forud for projektarbejdets begyndelse, har projektgruppen på baggrund af brainstorms og kravene i
studieordningen, udformet følgende problemformulering. \\

\textbf{Problem statement} \\
"How can the team design and implement a new universal booking system dedicated to movie theatres, 
that solves modern booking problems within this demanding industry?" \\

\textbf{Subquestions} \\
"How can the team through a system development method optimize the process and minimize errors along the way?" \\

"How can a software development approach improve the quality and produced work of the development team throughout this project?"\\

\section{Systemvision}
Eftersom der i dette semester projekt, ikke er tilknyttet en virksomhed er der blevet udarbejedet en Systemvision. 
Visionen har til formål at give indblik i biografsystemets formål, og er formuleret som følgende:

For members of the movie loving community, who appreciate having a collective application for booking 
in multiple locations and avoiding the disappointment of double booking. We have dedicated our 
time to develop the perfect platform to book movies and to offer an unique user experience and private user function. 
We strive to evolve and keep up with modern technology and demands from the customers. \\















% Here is the introduction. \cite{Mittelbach2005} \\

% Bilag kan ses i Bilag \ref{app:foerstebilag}.