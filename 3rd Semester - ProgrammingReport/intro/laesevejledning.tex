\newpage
\section*{Læsevejledning}
Rapporten er delt op i 9 kapitler, med dertilhørende undersektioner. Hvert kapitel indeholder et større segment af den samlede rapport.
Hvert kapitel begynder med en kort introduktion, der vil forklare hvad kapitlet vil forsøge at dokumentere. 
Til slut i hvert kapitel vil man kunne finde en delkonklusion.\\ 

Alle figurer og tabeller vil blive nummereret efter hvilket kapitel de optræder i. Det betyder for eksempel, at Figur 2.6 er den 
6. figur i Kapitel 2. \\

Kildehenvisninger er lavet efter IEEE, hvilket betyder, at [3] refererer til kilde nummer 3 i litteraturlisten. 
I litteraturlisten er bøger og artikler angivet med forfatter, tittel, forlag og årstal, mens internetsider er angivet med forfatter, tittel, dato og url. \\ 

I rapporten vil alle metodenavne blive skrevet med \textit{kursiv}, mens alle klasser og objekter vil blive skrevet med \textbf{fed}.\\

Ord som \textit{systemet} og \textit{programmet} vil blive brugt ofte i rapporten. Her refereres til det biografsystem, gruppen har udviklet i denne projektperiode.